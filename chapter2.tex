\chapter{A Survey of Gestural Interfaces}
\label{cha:survey}

\section{Definition}
\setcomment{We define gestural interfaces, and their various flavous, types and contexts here.}

\begin{todoenv}
What is a "gesture" in terms of a computer system?

These are defined according to basics texts on the subject. \cite{WixonWigdor2010}. Hand-based gestures and taxonomy as in \cite{Cassel1998}. 

Breakdown of general features, a gesture taxonomy and method of capture by classifiers

State-based, or "static" gestures versus "fluid" gestures, that encapsulate movement

Devices used in their capture, and the sort of gestures each captures
\end{todoenv}

\section{Intended Purpose and Origins}

\subsection{Natural Language Interaction}

\begin{todoenv}
Factoring from chapter 2

Gestures are a method of performing interaction with a computer system that mimics or copies the manner in which we communicate with other humans

This opens large possibilities for less direct and more ambiguous input when controlling and interacting with computer systems.

Discussing natural interaction, \cite{Cassel1998}. Use of existing sign languages (ASL, Kodaly), see: \cite{LicsárEtAl2006}, \cite{IrvingEtAl1975}, \cite{Woodward1976}.

The merge of natural language with interaction.

Spoken language as a modality (particularly \cite{Bolt1980}).

Uses in relation to tpyical tasks
English more prevalent in searching etc. and "matching" gestures to equivalent tasks (iconic representations). Consumer examples include Nintendo Wii, therapy papers on related topics also relevant.

Control of interfaces for motion relevant in diverse environments (see \cite{RoseEtAl2011}, \cite{DongwookEtAl2014}).

\end{todoenv}

\subsection{Expressiveness} 

\setcomment{Not a lot of material on this subject. To review.}

Limits in information transmission through unary inputs

Multi-dimensional addition given by gestures

Mulit-modality with gestural interfaces

Semantic richness; embedding large amounts of meaning into smaller sections of text.

\section{Development and Implementation}

This covers the history and development of the field of gestural interaction and input, exploring it form the earliest examples of body language capture to the modern devices used today.

\subsection{Two-dimensional / iconographic}

Historical perspective. The earliest gestural interfaces did very basic capture, either comparing images or one-dimensional input combined with other information for very basic input. Exploration should include very early pointing interfaces as handled by proximity sensors (\cite{Bolt1980}), Sketch pen interfaces (citation needed), and the like.

\subsection{Pen and tablet input}

\setcomment{Shorter Section.}
Looking at basic gestural interfaces that are caputred using tablet and pen, and their dvelopment. Early examples include Graffiti and optical character recognition interfaces. Move onto quick-type systems and towards the touch-based screens and devices we used today. The most unilateral and present form of gestural interface encountered by modern consumers.

\subsection{Motion sensing and towards "hands free"}

Earliest examples of camera solutions in computer science. Bulk of history can be explored here, due to the rapid rise in interests in the mid 80's and early 90's. Section should include camera efforts from early systems, trianguation, then moving towards depth cameras  and combined imputs. See endnote for citation list.

\subsection{Gesture Capture Devices}

Specialized capture devices, that mimic previous efforts. Discuss improvements in accuracy, particularly with the introduction of accelerometry. Includes data gloves, hand-hend controllers with sensors etc.

\section{Techniques in Recognition}

NOTE: This is probably the hardest section of the chapter. It's outside my field of confidence and is possibly an unnecessary edition. To discuss with supervisors, but a large body of work from ANU and beyond can be called on. Potentially a reference to \cite{CédrasShah1995} is sufficient.

