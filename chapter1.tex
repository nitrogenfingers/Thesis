\chapter{Natural Language and Gestures}
\label{cha:natural}

\setcomment{The introduction to the chapter should discuss natural gestures as clearly and succinctly as possible.}

\section{Inter-communicative gestures}
\label{sec:natural-intercom}

\setcomment{Introduce inter-communicative gestures as a concept. Define and give examples. Introduction and motivation for the chapter}

Gestures are a form of non-verbal communication that are performed largely subconsciously. They function as a vector of communication independently, or integrated with spoken language. The amount and kind of information gestures can convey depends on the form of gesture, but this can include things from basic context in spoken conversations, clarification or disambiguation of spoken concepts, conveying affective information and attitudes of the communicator, elaborating or demonstrating, particularly in reference to a physical place or task, or conveying information subtlety. The majority of inter-communicative gestures are performed by the hands and arms, and often in front of the body or around the face of the converser. Other aspects of indirect gestural communication, such as posture and gait can also affect this information.

This thesis begins with an examination of the natural occuring context of gestures to better understand their role in basic human communication, and how best they would intergrate with human-computer communication.

Sources: \cite{Argyle1988}, \cite{KnappHall2010}, \cite{Cassell1998}.

\begin{todoenv}

\end{todoenv}

\section{Forms of inter-communicative gestures}
\label{subsec:natural-forms}
\setcomment{Discuss the differences between iconic, deictic, metaphoric and beat gestures as features of inter-communicative language, as well as ergotic gestures used in human-computer interaction}

Four forms of inter-communicative gesture: Iconic, deictic, metaphoric and beat gestures. These are all used when integrated with human speech, although some can be used externally.

Iconic gestures are also known as 'miming gestures'. These involve performing hand or body movements that are a facsimilie or equivalent to the gestures performed in doing a specific task with a given object or in a certain environment. The gesture clarifies to the observer what the action is, or may point out how the action is done, what it involves etc.

Metaphoric gestures are specifically inter-communicative and are used to reference conversational objects or topics as though they had a physical presence. This is mostly used to disambiguate non-declarative language; where the indefinite is used (e.g. 'I was talking to him') a metaphoric gesture may point to a previously referenced physical space to clarify.

Deictic, or pointing gestures serve the purpose of associating a physical object or direction with the topic of discourse to the interlocutor.

Beat gestures are naturally-occuring with spoken language, are broadly univerasally performed (with variation, see \cite{Argyle1988} ) as a means of adding emphasis and stress to words or sections. This allows certain sections that are peripheral or unimportant to be skipped, while the most important points are better emphasized.

Emblematic gestures can be considered a subset of iconic gestures, or may exist on their own; these are gestures that have an innate (typically simple) meaning, typically indicating commands for the interlocutor or very simple attitudes. Sets of these typically exist in all languages; and though they may have evolved from a specific event or feature of the natural world they are now broadly arbitrary. Complex syntaxes of emblematic gestures can form their own languages; such is the case for code signs used by police and sign language used by the deaf.

\section{Elements of gestural communication}
\label{sec:natural-elems}

\setcomment{A discussion of all elements involved in secondary communication. Includes prevalence and significance of using gestures to communicate, as well as the hand and body movements made, or other features (like eye contact, or simpler/subtler movements}

\begin{todoenv}

\end{todoenv}

\section{Cultural variation and significance}
\label{sec:culture-variation}

\setcomment{Covers a number of references exploring gestural performance and variation between different cultural groups, including separate countries, sign language communicators, and broader groups. Contrasting both similarities and differences}

Gestures are a universal component of language; no living language is known to explicitly exclude their usage, and they are generally thought to provide additional information to spoken language, making it better to understand and easier to follow.

Gestures are observed in non-human species, particularly primates. Many breeds of monkey have been observed manipulating their hands and pose to express different information, particularly affective information about their mood or their relationship to other monkeys (submissive, dominant, sexual etc.). Such expressive gestures can involve hand, head or full body movements, often accompanied with cries or a specifically adopted posture. Although uncommon, monkeys have been observed using pointing gestures, and many animals are capable of understanding humans when pointing. \cite{Goodall1968} \cite{Argyle1988}

Gestural language is considered a universal feature of human communication. It has been hypothesized that gestures may have been one of the earliest modes of communication, developing before or concurrently with speech, given it has been shown breeds of gorilla and chimpanzee are able to learn and use sign language competently without further evolution, while the same is not true for vocal communication. \cite{Hewes1976}. Some forms of gestures can be found in all living languages, and there exists a small subset of gestures that are considered linguistically universal; these are recognized and used in all cultures with a reasonably similar meaning; it is expected such gestures may have evolved from natural behaviour or very early human interaction. Many such gestures however have although the same performance can have completely different interpretations depending on the culture \cite{KnappHall2010}. This can mean that gestures not only have many different meanings depending on the language or region they are performed, but they can have very different meanings; for example the thumbs up, which is a simple acknowledgement in most languages, can be highly derogatry or offensive in others.

It can be expected that all computer users proficient in a natural language will have no difficulty understanding how gestures may be performed to operate a computer system, especially if they are already familiar with computer systems or it is presented in the context of typical human communication (see \cite{Cassel1988}). However, a single set of gestures will not be sufficient for a broad user base, especially if that base is multi-cultural due to the very high level of variation. Certain gestures may be unintuitive or complicated to the performer as they have no cultural significance, others may find them uncomfortable to perform due to cultural connotations or illogial given the circumstances. A common example is the direct deictic point; although a simple and very common gesture in both spoken language and computer usage, pointing with the index finger at someone can be considered highly rude in some cultures, limiting the utility of this gesture in certain situations without forcing users to behave unnaturally.

\section{Variation in performance of gestures}
\label{sec:natural-variation}

Gestures are a part of all languages, though are performed different and mean different things between cultures. The question of whether or not they are performed differently by different groups of individuals or even if gestural performances remain consistent with an individual marks at the core of what 'comprises' a gesture. Most gestures are defined by the general motion made or the positioning relative to other parts of the body, where as recognition is often more specific. Isolating what exactly people naturally think of in defining gestures, and how that might change the way they perform them over time, would help us to better understand how to adapt to users over many repeated performances, as well as how to define our systems in general.

This will be discussed in more detail in chapter 4.

\fix{This section is currently undercited. More results can be found to fill it out.}

\setcomment{This is moving towards the research done in experiment (discussed in chapter 4 in more detail). Potentially worth moving to a later slot.}
\fix{relabel}

%Does this merit a chapter? Do we discuss communication between parties
%sufficiently often for this to be a valuable exercise for the reader


%At the begging of each chapter, please introduce the motivation and high-level
%picture of the chapter. You also have to introduce sections in the
%chapter. \\
%Section~\ref{sec:motivation} xxxx.\\
%Section~\ref{sec:relatedwork} yyyy.\\
%\section{Motivation}
%\label{sec:motivation}
%\section{Related work}
%\label{sec:relatedwork}
%You may reference other papers. For example: 
%Generational garbage collection~\citep{LH:83,Moon:84,Ungar:84} is perhaps the
%single most important advance in garbage collection since the first collectors
%were developed in the early 1960s. (doi: "doi" should just be the doi part, not
%the full URL, and it will be made to link to dx.doi.org and resolve.
%shortname: gives an optional short name for a conference like PLDI '08.)
%\section{Summary}
%Summary what you discussed in this chapter, and mention the story in next
%chapter. Readers should roughly understand what your thesis takes about by only reading
%words at the beginning and the end (Summary) of each chapter.