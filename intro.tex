\chapter{Introduction}
\label{cha:intro}

\begin{commentenv}
The introduction will define the topic of the thesis. It's goal is to provide context to the reader for the information presented in the rest of the body of the work. It is non-technical, gestalt and comprehensive.
\end{commentenv}

\section{Topic}
\label{sec:topic}

\setcomment{The topic provides an initial discussion into what the thesis is about on the whole. It provides the first introduction to the topic from a simplified perspective, and how the thesis addresses the topic matter.}

\begin{todoenv}
This thesis explores the topic of interface that makes use of human hand and body movements as a modality of input, referred to as gestural interfaces. Such interfaces, which measure this movement either thorugh the use of sensors attached to the body or through cameras, allow humans to interact with interfaces either in a way that better represents the task they are trying to perform, such as kicking a ball in a virutal environment, pushing a 3D object or pointing to an area of the screen to interact with directly. This perception of ease of use combined with the opportunity for more expressive input and novelty factor have seen gestural interfaces becoming commonplace consumer devices.

However, with this increase in consumer interests, more and more attention is drawn to the drawbacks of touchless gestural interfaces. Limitations in the technology in terms of gesture capture and classification mean such interfaces can be difficult for individuals to use; gestures are typically defined by a very rigid requirement for movement which can be difficult to replicate, especially if the movement required is not communicated sufficiently. Even then, performance of such interfaces is never 100\% and given the smaller set of interactions performed and the higher time and effort investment required to perform them, this lower recognition rate can detract from the user experience negatively. Ergonomics are also an increasingly prevalent concern; most computer input requires low physical movement and effort to operate, and the higher requirement for physical movement can fatigue users. Poorly defined gestures that demand repetitive movement, hyperflexion or unnatural joint movement can also lead to strain and discomfort even in short sessions of use.

These issues are sufficiently prevalent that work into finding more ergonomic, comfortable and useful gestures for typical computer usage has become the predominant focus of many such technologies in the past few years. This thesis intends to add to the body of knowledge on the subject by exploring factors specific to how humans interact with such interfaces; specifically treating such interfaces as natural, or inherent in their own behaviour and observing how they would interact in a low-constraints environment. The ultimate goal of this research is to find a set of guidelines or constraints to match as many users as possible, or find where flexibility is necessary in gestural interfaces, with the ultimate goal of making them both more comfortable and more intuitive to use in future iterations.
\end{todoenv}

\section{Research Question}
\label{sec:researchquestion}

\setcomment{This discusses in detail the questions to be answered in this thesis}

\begin{todoenv}
We wish to answer the following questions in this research:

1. Can a user interact with a gestural interface naturally? When given a low-constraints system, the natural interface hypothesizes there is a natural method in which users will wish to interact given how the display is provided and the obvious system constraints made visible, such as environmental features (controllers, or how the display presents the information shown). If given a loosely-constrained system, what gestural interactions would a user naturally use? Are there limitations or requirements that enforce this?

2. How do users reconcile fatigue with interaction? Such interfaces typically produce significant fatigue. How would the user adjust their performance to avoid this, if at all? How can we make our systems reactive of this?

3. How do users prefer to interact with gestural systems? Do they prefer a highly-constrained environment or a more relaxed environment? What kind of gestures do they most like to perform when doing typical tasks? Do we see abstraction or highly iconic gestures being preferred for most interactions?

\end{todoenv}

\section{Thesis Outline}
\label{sec:outline}

\setcomment{The manner in which the research was conducted and how it is presented is discussed in this section.}

\begin{todoenv}
Chapter~\ref{cha:natural} of this text describes gestures as a vector of human-human communication, and some of these applications in the realm of human-computer interaction. This includes the structure and syntax of natural gestures, their purpose in communication, the similarities and differences in cultures and groups. It also touches on the use of direct, natural languages like ASL and Kodaly notation in computer usage.

Chapter~\ref{cha:survey} is a review of touchless gestural interfaces. It moves from early gestural interfaces, taking a historical perspective and analysing key features and purposes of systems found in the literature, before systematically comparing systems to quantify function and utility of such systems in contextualizing the use-cases this thesis is exploring.

Chapter~\ref{cha:implementation} discusses the design and implementation of a deictic gestural interface. The context of the system design and purpose is explained, as well as how it was used. Issues with implementation and lessons learned from the design are also covered as background material. This is all presented for discussion in the next chapter.

Chapter~\ref{cha:pointing} describes the experiments run using the gestural control interface. It covers a series of experiments that involve variations on the interface, controlling various aspects of how the participants interact. Results are presented with discussion.

Chapter~\ref{cha:freeform} describes another set of experiments run exploring users with a freely-defined gestural interface, and variations on that set of experiments. Results are presented as before.

Chapter~\ref{cha:conclusion} discusses the results in the context of the reseach question. It explains their significance and extrapolates on how they can be applied and their value in designing gestural interfaces for user interaction, and concludes the work.

There are two appendices: Appendix A describes the application of the pointing interface to a data visualization, while Appendix B explores their value in human-robotic interfaces.
\end{todoenv}


\section{Significance}
\label{sec:significance}

\setcomment{Discusses the impact of the work in the context of the problem and existing research}
\fix{Potentially a topic for the conclusion rather than introduction- discuss with supervisor}